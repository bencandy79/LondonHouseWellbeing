% Chapter Template

\chapter{Data Sources} % Main chapter title

\label{Chapter3} % Change X to a consecutive number; for referencing this chapter elsewhere, use \ref{ChapterX}

%----------------------------------------------------------------------------------------
%	SECTION 1
%----------------------------------------------------------------------------------------

\section{Flat files}

Lorem ipsum dolor sit amet, consectetur adipiscing elit. Aliquam ultricies lacinia euismod. Nam tempus risus in dolor rhoncus in interdum enim tincidunt. Donec vel nunc neque. In condimentum ullamcorper quam non consequat. Fusce sagittis tempor feugiat. Fusce magna erat, molestie eu convallis ut, tempus sed arcu. Quisque molestie, ante a tincidunt ullamcorper, sapien enim dignissim lacus, in semper nibh erat lobortis purus. Integer dapibus ligula ac risus convallis pellentesque.

%-----------------------------------
%	SUBSECTION 1
%-----------------------------------
\subsection{London datastore}

Nunc posuere quam at lectus tristique eu ultrices augue venenatis. Vestibulum ante ipsum primis in faucibus orci luctus et ultrices posuere cubilia Curae; Aliquam erat volutpat. Vivamus sodales tortor eget quam adipiscing in vulputate ante ullamcorper. Sed eros ante, lacinia et sollicitudin et, aliquam sit amet augue. In hac habitasse platea dictumst.

%-----------------------------------
%	SUBSECTION 2
%-----------------------------------

\subsection{Transport for London}
Morbi rutrum odio eget arcu adipiscing sodales. Aenean et purus a est pulvinar pellentesque. Cras in elit neque, quis varius elit. Phasellus fringilla, nibh eu tempus venenatis, dolor elit posuere quam, quis adipiscing urna leo nec orci. Sed nec nulla auctor odio aliquet consequat. Ut nec nulla in ante ullamcorper aliquam at sed dolor. Phasellus fermentum magna in augue gravida cursus. Cras sed pretium lorem. Pellentesque eget ornare odio. Proin accumsan, massa viverra cursus pharetra, ipsum nisi lobortis velit, a malesuada dolor lorem eu neque.

%-----------------------------------
%	SUBSECTION 3
%-----------------------------------

\subsection{gov.uk}
Morbi rutrum odio eget arcu adipiscing sodales. Aenean et purus a est pulvinar pellentesque. Cras in elit neque, quis varius elit. Phasellus fringilla, nibh eu tempus venenatis, dolor elit posuere quam, quis adipiscing urna leo nec orci. Sed nec nulla auctor odio aliquet consequat. Ut nec nulla in ante ullamcorper aliquam at sed dolor. Phasellus fermentum magna in augue gravida cursus. Cras sed pretium lorem. Pellentesque eget ornare odio. Proin accumsan, massa viverra cursus pharetra, ipsum nisi lobortis velit, a malesuada dolor lorem eu neque.

%----------------------------------------------------------------------------------------
%	SECTION 2
%----------------------------------------------------------------------------------------

\section{APIs}

A subset of London data is also available via various APIs. For transport information Transport for London have extensive APIs available, unfortunately none of these were able to provide any of the datasets required for the project. The use of APIs became focussed on collecting information on venues to obtain the data for the 'community vitality' and 'access to cultural spaces' datasets. For this data Foursquare was the primary candidate with venue information available with no cost implications. 

%-----------------------------------
%	SUBSECTION 1
%-----------------------------------
\subsection{Foursquare}

Nunc posuere quam at lectus tristique eu ultrices augue venenatis. Vestibulum ante ipsum primis in faucibus orci luctus et ultrices posuere cubilia Curae; Aliquam erat volutpat. Vivamus sodales tortor eget quam adipiscing in vulputate ante ullamcorper. Sed eros ante, lacinia et sollicitudin et, aliquam sit amet augue. In hac habitasse platea dictumst.

%-----------------------------------
%	SUBSECTION 2
%-----------------------------------

\subsection{Google Maps}
The places API from Google Maps was queried and code written to extract venue data. However, after obtaining some basic search results it was decided that Google's pricing policy and rate limits made this data source unfeasible for the scope of the project.

%----------------------------------------------------------------------------------------
%	SECTION 3
%----------------------------------------------------------------------------------------

\section{Data availability}

In part thanks to the Greater London Authority’s Datastore, a wealth of information is available at London Ward level. Combining this with information from the Foursquare API and Department for Education schools data provided a significant body of information in which to model London as a multiple dataset.
Data availability was such that datasets for each of the indicators feeding into the six domains were available and it was not necessary to use substitute indicators that differed significantly from those originally planned. 

%-----------------------------------
%	SUBSECTION 1
%-----------------------------------
\subsection{Timeliness}

Some of the datasets related to different years, often linked to census years and administrative or electoral changes.
Wherever possible the most recent data has been used to synchronise as closely as possible with the 2017 data used for the median house price data. For example, Emission data is from 2011 as this is the most recently published at ward level. The London Air Quality daily feed run by King’s College does not have enough coverage to sufficiently differentiate over 600 wards. Ideally this data would have been from the same year as the house price information.

%-----------------------------------
%	SUBSECTION 2
%-----------------------------------

\subsection{Boundary changes}
Ward boundaries were re-drawn in three London Boroughs in 2014 meaning that a function had to be written that, for data pre-dating 2014, would map old ward codes to the new ward that contained the largest section of the newly defined area. Newer data would also have been helpful here but the commonality of area between the old and new codes in the mapping should ensure that the data is representative of the new area.

%-----------------------------------
%	SUBSECTION 3
%-----------------------------------
\subsection{API rate limits}

API limits and costs imposed limits on the depth of information that could be obtained from these sources. Whilst I was able to obtain the necessary Foursquare venue and venue location information, venue ratings and check-ins were subject to stringent daily limits which meant that obtaining this information would have taken weeks of daily iterations or significant costs neither of which were feasible for this project. I investigated the possibility of using Google Places API but the new pricing model introduced this year also made this unfeasible. In a commercial setting it may be possible to increase the amount of data that can be obtained from some of the APIs.
