% Chapter Template

\chapter{Data Sources} % Main chapter title

\label{Chapter3} % Change X to a consecutive number; for referencing this chapter elsewhere, use \ref{ChapterX}

%----------------------------------------------------------------------------------------
%	SECTION 1
%----------------------------------------------------------------------------------------

\section{Datasets required to create the well-being indices}

To create the well-being indices the decision was made to use six indices each made up of three indicator datasets. For the purpose of the project, each index and indicator will be of equal weighting.
This approach is based upon other indices of well-being where data feeds into a set of between five and ten indices. 
Three examples of this type of index would be the Gross National Happiness index (\cite{GNH}), the OECD Better Life Initiative (\cite{BLI}) and the Canadian Index of Wellbeing (\cite{CIW}). 
The approach taken was to find three indicators that both represented each index and where the data could be retrieved or transformed to be analysed at ward level.

\begin{table}[H]
\caption{Indices and indicators used for the measurement of well-being for each London ward with data source and metric used}
\scalebox{0.4}{
\begin{tabular}{llll}
\toprule
Index                    & Indicators  & Source & Metric       \\
\midrule
\multirow{3}{*}{Education and Employment} & Average GCSE points & London datastore & Average GCSE capped point score \\
                         & Average Ofsted rating of schools & edubase @ gov.uk & Average Ofsted ratings of schools - most recent inspections (September 2018) \\
                         & Employment rates  & London datastore   & Rate of economically active individuals (2011)            \\
\hline
\multirow{3}{*}{Safety and Security}     & Crime rates - against the person  & London datastore & Number of recorded crimes (2017)\\
                         & Crime rates - other crime    & London datastore  & Number of recorded crimes (2017) \\
                         & Traffic injuries  & London datastore & Road collision numbers (2014) \\
\hline
\multirow{3}{*}{Environment}     & Air pollution & London datastore & Annual mean of Nitrogen Dioxide (NO2) and Particle emissions (PM10) (2011) \\
                         & Amount of greenspace  & London datastore    & Percentage of ward area which is greenspace (2011)  \\
                         & Access to nature & London datastore & Percentage of residential households with access to at least one open space (2011)\\
\hline
\multirow{3}{*}{Community Vitality and Participation} & Access to cultural space  & Foursquare & Number of cultural venues in ward (August 2018)\\
                         & Amount of bars and restaurants   & Foursquare  & Number of bars and restaurants in ward (August 2018)\\
                         & Election turnout & London datastore  & Percentage of residents voting in 2016 London election\\
\hline
\multirow{3}{*}{Infrastructure} & Access to public transport & London datastore & Transport for London Public Transport Accessibilty Levels (2015) \\
                         & Average journey times   & gov.uk  & The mean percentage value of those within an hour's travel by public transport or walking of an area of 100, 500 and 500 jobs (2014)\\
                         & Population density & London datastore & Persons per square km (2013)\\
\hline
\multirow{3}{*}{Health}  & Life expectancy & London datastore & Life expectancy at birth (2013) \\
                         & Childhood obesity   & London datastore   & Prevalence of obese children at age 11 (2013)  \\
                         & Illness preventing employment & gov.uk  &  Comparative illness and disability ratio from English Indices of Deprivation (2015)\\
\bottomrule\\              
\end{tabular}
}
\end{table}

Access to open space definition \footnote{\url{https://data.london.gov.uk/dataset/access-public-open-space-and-nature-ward}}

Access to public transport definition \footnote{\url{https://data.london.gov.uk/dataset/public-transport-accessibility-levels}}

Illness preventing employment definition  \footnote{\url{https://www.gov.uk/government/statistics/english-indices-of-deprivation-2015}}

%----------------------------------------------------------------------------------------
%	SECTION 2
%----------------------------------------------------------------------------------------

\section{Static data files}

The majority of data sets are either csv or Excel files which are accessed directly via a url by the data importer application. These sources are primarily from the websites of public bodies. The data importer application makes use of the pandas module within Python to support the importing and cleaning process within a series of dataframe objects.

%-----------------------------------
%	SUBSECTION 1
%-----------------------------------
\subsection{London datastore}

The Greater London Authority Datastore (\cite{GLA}) was the primary source of data for the project with significant amounts of data relating to various London metrics available. The fact that much of the data contained on the website was available at ward level was key in being able to piece together the indices.


%-----------------------------------
%	SUBSECTION 2
%-----------------------------------

\subsection{gov.uk}

Used for Ofsted ratings for schools, average journey times and illness affecting employment, the gov.uk website contains a varied range of datasets. 
Three key datasets used from this source are Edubase schools information for Ofsted ratings, 2014 Department for Transport journey time statistics and 2015 English Indices of deprivation.
Without the London focus of the GLA datastore, information on gov.uk is often harder to locate and in many cases is at a higher level of aggregation than required with most information at borough or national level.

%-----------------------------------
%	SUBSECTION 4
%-----------------------------------

\subsection{esri}

Data for average journey times and illness affecting employment was obtained at lower super output area, a lower level of aggregation than ward. The different geographical level use in the two datasets created the need for a mapping. The open data section of esri's ArcGIS website (\cite{ArcGIS}) offered this mapping document in csv format which could be incorporated into the data import software to allow an arithmetic mean to be determined at ward level.


%----------------------------------------------------------------------------------------
%	SECTION 3
%----------------------------------------------------------------------------------------

\section{APIs}

A subset of London data is also available via various APIs. For transport information Transport for London have extensive APIs available, unfortunately none of these were able to provide any of the datasets required for the project, though other formats offered by TFL were used. The use of APIs became focussed on collecting information on venues to obtain the data for the 'community vitality' and 'access to cultural spaces' datasets. For this data Foursquare was the primary candidate with venue information available with no cost implications. 

%-----------------------------------
%	SUBSECTION 1
%-----------------------------------

\subsection{Foursquare}

Foursquare is a social media platform based on location intelligence. The platform has information on more than 105 million locations mapped worldwide and over 50 million users per month. (\cite{foursquare}). With the extensive database of venues with location information, Foursquare could be used to look at two of the indicators which feed into the Community Vitality and Participation index:
\begin{itemize}
\item Access to cultural spaces
\item Popularity of bars and restaurants
\end{itemize}
To measure these two indicators, Foursquare's venue API can be used to return list of venues within a radius of a specific point provided in latitude and longditude.
By using the three venue types, "culture", "food" and "nightlife", a list of venues fitting those types can be captured via the API. In the implementation phase, the list of venues containing the point location of the venues can be mapped to the relevant ward and numbers aggregated to give an indicator of volume with which to measure this category.


%-----------------------------------
%	SUBSECTION 2
%-----------------------------------

\subsection{Google Maps}
The places API from Google Maps was queried and code written to extract venue data. However, after obtaining some basic search results it was decided that Google's pricing policy and rate limits made this data source unfeasible for the scope of the project.

%----------------------------------------------------------------------------------------
%	SECTION 3
%----------------------------------------------------------------------------------------

\section{Shape Files}

For choropleth mapping, shape files are needed to visualise the polygon shapes which define adminstrative or political boundaries. Shape files usually consist of properties of the shape such as name and size along with a geometry feature which defines the boundaries of a given shape via the coordinates of their vertices. For this project, the relevant shapefile is key to creating the final interactive map and will also be use to visualise the data and support the model building process.  

%-----------------------------------
%	SUBSECTION 1
%-----------------------------------
\subsection{London Ward shapefile}

The key shapefile for the project is taken from the London datastore and provides the polygon geometry of the London wards. This is a .shp file where the point coordinates which make up the polygon for each ward are in the Ordnance Survey grid reference cooridnate reference system. With point geometries more often listed in latitude and longditude, there will need to be some reprojection of coordinates to enable certain datasets to match with the shapefile.

%----------------------------------------------------------------------------------------
%	SECTION 4
%----------------------------------------------------------------------------------------

\section{Data availability issues}

In part thanks to the Greater London Authority’s Datastore, a wealth of information is available at London Ward level. Combining this with information from the Foursquare API and Department for Education schools data provided a significant body of information in which to model London as a multiple dataset.
Data availability was such that datasets for each of the indicators feeding into the six indices were available and it was not necessary to use substitute indicators that differed significantly from those originally planned. 

%-----------------------------------
%	SUBSECTION 1
%-----------------------------------
\subsection{Timeliness}

Some of the datasets related to different years, often linked to census years and administrative or electoral changes.
Wherever possible the most recent data has been used to synchronise as closely as possible with the 2017 data used for the median house price data. For example, Emission data is from 2011 as this is the most recently published at ward level. The London Air Quality daily feed run by King’s College does not have enough coverage to sufficiently differentiate over 600 wards. Ideally this data would have been from the same year as the house price information.

%-----------------------------------
%	SUBSECTION 2
%-----------------------------------

\subsection{Boundary changes}
Ward boundaries were re-drawn in three London Boroughs in 2014 meaning that a function had to be written that, for data pre-dating 2014, would map old ward codes to the new ward that contained the largest section of the newly defined area. Newer data would also have been helpful here but the commonality of area between the old and new codes in the mapping should ensure that the data is representative of the new area.

%-----------------------------------
%	SUBSECTION 3
%-----------------------------------
\subsection{API rate limits}

API limits and costs imposed limits on the depth of information that could be obtained from these sources. Whilst I was able to obtain the necessary Foursquare venue and venue location information, venue ratings and check-ins were subject to stringent daily limits which meant that obtaining this information would have taken weeks of daily iterations or significant costs neither of which were feasible for this project. I investigated the possibility of using Google Places API but the new pricing model introduced this year also made this unfeasible. In a commercial setting it may be possible to increase the amount of data that can be obtained from some of the APIs.
