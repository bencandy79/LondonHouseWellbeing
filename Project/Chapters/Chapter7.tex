% Chapter Template

\chapter{Conclusions and evaluation} % Main chapter title

\label{Chapter7} % Change X to a consecutive number; for referencing this chapter elsewhere, use \ref{ChapterX}

%----------------------------------------------------------------------------------------
%	SECTION 1
%----------------------------------------------------------------------------------------

\section{Lessons learnt}


%-----------------------------------
%	SUBSECTION 1
%-----------------------------------

\subsection{Summary of findings}



%----------------------------------------------------------------------------------------
%	SECTION 2
%----------------------------------------------------------------------------------------

\section{Possible developments}

An extension of this project that would be an interesting addition to the results already displayed would be to look at whether changes in the well-being indices were related to changes in house prices. Such analysis could potentially provide insight on whether increasing house prices fuel rises in well-being indicators or whether improvements in local well-being increase demand for houses in a particular area. Correlation here would support causation, either in the case of increased demand for houses where "wellness" is high or that when house prices rise, an area will gentrify and well-being indicators will rise as a result ? THIS NEEDS A CITATION?
To achieve this, more data would need to be found that could represent all of the well-being indicators for regular intervals over a significant period of time. Given that some of the datasets are only released every few years, this may require looking at some innovative solutions of data collection with crowdsourcing and image analysis possible avenues of investigation. For instance, recent work around the analysis of social media photos (? CITE HIPSTER RESEARCH ?) has created a model to predict gentrification of city areas based on the presence of happiness (smiling people, cultural events) in such images.
Another area to look at would be how the flow of residents, particularly between neighbouring wards, can affect the house prices. For instance, if the perception of a particular area declines or potentially undesirable changes occur (eg. large construction projects), how does the inflow and outflow affect the prices. This may be an area where the use of complex networks could assist with analysis.
If being run as a commercial application, a paying subscription to the Foursquare and/or Google API would allow further analysis around assessing venue popularity to feed into the Community Vitality and Participation domain.

